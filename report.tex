\documentclass[paper=a4, fontsize=11pt]{scrartcl}
\usepackage[protrusion=true,expansion=true]{microtype}	
\usepackage{url}

%%% Custom sectioning
\usepackage{sectsty}
\allsectionsfont{\centering \normalfont\scshape}


%%% Custom headers/footers (fancyhdr package)
\usepackage{fancyhdr}
\pagestyle{fancyplain}
\fancyhead{}											% No page header
\fancyfoot[L]{}											% Empty 
\fancyfoot[C]{}											% Empty
\fancyfoot[R]{\thepage}									% Pagenumbering
\renewcommand{\headrulewidth}{0pt}			% Remove header underlines
\renewcommand{\footrulewidth}{0pt}				% Remove footer underlines
\setlength{\headheight}{13.6pt}


%%% Maketitle metadata
\newcommand{\horrule}[1]{\rule{\linewidth}{#1}} 	% Horizontal rule

\title{
        %\vspace{-1in} 	
        \usefont{OT1}{bch}{b}{n}
        \normalfont \normalsize \textsc{University of California, Berkeley} \\ [5pt]
        \horrule{0.5pt} \\[0.4cm]
        \huge Finding optimal kidney exchange cycles \\
        \horrule{2pt} \\[0.5cm]
}
\author{
        \normalfont 								\normalsize
        Philip Hayes, Tommy Anthony, Arani Bhatta\\[-3pt]		\normalsize
        SIDs: 25394273, 25280364, 25280364 \\[-3pt]\normalsize
        \today
}
\date{}


%%% Begin document
\begin{document}
\maketitle

We created 4 different algorithms, each of which covered different areas of the problem
space. The primary motivation for the ensemble of algorithms was the wide range of
graph densities in our instance set; thus, our objective for each algorithm was to target a
specific subset of the problem space. We also took advantage of the highly optimized
mixed linear integer programming (MILP) library, Gurobi \cite{Gurobi15}.

\paragraph{}

The first step in solving an instance was splitting the graph into its strongly connected
components, each of which could be solved independently. For instances with densities
below 25\%, we designed a MILP which copied the graph $|V|$ times and made the restriction
that the $l^{th}$ copy of the graph contained only a single cycle beginning and ending at
the $l^{th}$ vertex. We then enforced flow and cycle length constraints across every graph
copy. This solved most medium-low density graphs to optimality. This algorithm was inspired
by the polynomial-sized Extended Edge model presented in Constantino et al. (2013)
\cite{Constantino13}.

\paragraph{}

For graphs of less than 40\% density, we found that the optimal $k\le 3$-cycle allocation
achieved optimality in much less time than the previously described algorithm. Here, we exhaustively
enumerated every possible $k\le 3$-cycle, formulated it as a MILP optimization, and enforced vertex
constraints.

\paragraph{}

For the highly dense graphs of greater than 40\% density, we created two different algorithms.
The first was a $k=2$-cycle MILP which reinterpreted the directed graph as a purely undirected
graph and then optimized over the set of undirected edges (2-cycles). The second was a randomized,
greedy cycle finder inspired by the description presented in Abbassi et al. (2008) \cite{Abbassi08}
that achieved better scores than the $k=2$-cycle cover on select instances.

\newpage

\begin{thebibliography}{9}

\bibitem{Gurobi15}
    Gurobi Optimization, Inc.
    "Gurobi Optimizer",
    \url{http://www.gurobi.com},
    2015.

\bibitem{Constantino13}
    Constantino, Miguel, et al.
    "New insights on integer-programming models for the kidney exchange problem."
    \emph{European Journal of Operational Research}
    231.1 (2013): 57-68.

\bibitem{Abbassi08}
    Abbassi, Zeinab, and Laks V. S. Lakshmanan.
    "Offline Matching Approximation Algorithms in Exchange Markets."
    \emph{Proceeding of the 17th International Conference on World Wide Web - WWW'08}
    (2008): Web.

\end{thebibliography}

\end{document}
